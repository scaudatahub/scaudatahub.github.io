%Example of use of oxmathproblems latex class for problem sheets
\documentclass{oxmathproblems}

%(un)comment this line to enable/disable output of any solutions in the file
%\printanswers

%define the page header/title info
\oxfordterm{SP21}
\course{Probability}
\sheetnumber{1}
\sheettitle{第一次作业} %can leave out if no title per sheet

% add further contact details to footer if desired,
%e.g. email address, or name and email address
\contact{Zhu Fang, zhufang147@gmail.com}


\begin{document}

\textcolor{red}{
\section*{作业说明}
\begin{enumerate}
  \item 请在原模板的题目下面进行解答,建议使用proof环境,具体请使用搜索引擎.
  \item 题目有些是英文有些是中文,有时会给出教材的章节数和题号.
  \item {解答请统一使用中文作答,课本内的题请标好题号,如:第一单元$Problems$部分第$1$题:
  \[ \text{Unit 1 Problem 1 }. \]
  }
  \item 较难的习题在课本上都有答案,请仔细思考后再参考,随后按照自己的思路作答.
  \item 作业打分按照完成度进行打分,对错不严格要求.
  \item 请在规定时间内上交作业,具体细节请参考课程主页.
\end{enumerate}
}
\newpage



\begin{questions}

\miquestion 利用课堂上所学习的概率论公理证明下列推论:
\begin{parts}
  \part If $ A \subset B$, then $P(A) \leq P(B)$
  \part $P(A \cup B) = P(A) + P(B) - P(A \cap B)$
  \part $P(A \cup B) \leq P(A) + P(B)$
  \part $P(A \cup B \cup C) = P(A) + P(A^c \cap B) + P(A^c \cap B^c \cap C)$
\end{parts}

\miquestion
考虑指标集合$I$,已知对任意的$i, j \in I$,都有$A_i \cap A_j = \emptyset$,问:
\[ P(\sum_{k \in I}A_k )= \sum_{k \in I} P(A_k)\]
是否一定成立?若成立请给出证明,若不成立,请举出反例.

\miquestion
教材第一单元Sample Space and Probability 习题部分:
\[ 1\sim 4, 6, 9, 11\sim 20, 22, 24, 25, 26, 28, 31, 38, 40, 42\sim 48 \]

\end{questions}
\end{document}
