%Example of use of oxmathproblems latex class for problem sheets
\documentclass{oxmathproblems}

%(un)comment this line to enable/disable output of any solutions in the file
%\printanswers

%define the page header/title info
\oxfordterm{SP21}
\course{Linear Algebra}
\sheetnumber{1}
\sheettitle{第一次作业} %can leave out if no title per sheet

% add further contact details to footer if desired,
%e.g. email address, or name and email address
\contact{Zhu Fang, zhufang147@gmail.com}


\begin{document}

\textcolor{red}{
\section*{作业说明}
\begin{enumerate}
  \item 请在原模板的题目下面进行解答,建议使用proof环境,具体请使用搜索引擎.
  \item 题目有些是英文有些是中文,有时会给出教材的章节数和题号.
  \item 解答请统一使用中文作答.
  \item 作业打分按照完成度进行打分,对错不严格要求.
  \item 请在规定时间内上交作业,具体细节请参考课程主页.
\end{enumerate}
}
\newpage



\begin{questions}

\miquestion 
给定三维空间中的向量$\bm{x}, \bm{y}, \bm{z}$,其线性组合$a\bm{x} + b\bm{y} + c \bm{z}$一定能充满整个三维空间吗?
若能请给出证明,若不能请给出反例.

\miquestion
证明:
$$
\sqrt[3]{xyz} \leq \frac{1}{3} (x + y + z).
$$

\miquestion
证明:对角占优的矩阵是可逆矩阵,其中对角占优指的是:
$$
|a_{ii}| > \sum_{j \neq i } |a_{ij}|.
$$
\miquestion
给定可逆矩阵$\bm{A}$,请问$\bm{A}^{-1}$唯一吗?试给出证明.
\miquestion
\begin{itemize}
  \item Section 1.2 Problem 30.
  \item Section 2.3 Problem 3, 8, 18, 25, 29, 30, 31.
  \item Section 2.4 Problem 6, 15, 16, 17, 18, 21, 36, 37, 38.
  \item Section 2.5 Problem 10, 16, 18, 22, 24, 25, 28, 29, 30, 32, 39, 43, 44.
\end{itemize}



\end{questions}
\end{document}
