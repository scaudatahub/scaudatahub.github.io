%Example of use of oxmathproblems latex class for problem sheets
\documentclass{oxmathproblems}

%(un)comment this line to enable/disable output of any solutions in the file
%\printanswers

%define the page header/title info
\oxfordterm{SP21}
\course{Probability}
\sheetnumber{0}
\sheettitle{自查作业} %can leave out if no title per sheet

% add further contact details to footer if desired,
%e.g. email address, or name and email address
\contact{Zhu Fang, zhufang147@gmail.com}


\begin{document}

\begin{questions}

\miquestion 试通过观察牛顿二项式公式:
  
\[ 
  (1+x)^n = C_{n}^{0}  + C_{n}^{1} x + \cdots + C_{n}^{n} x^n
\]
来证明下列几个组合恒等式:
\begin{parts}
  \part $C_{n}^{0}  + C_{n}^{1}  + \cdots + C_{n}^{n} = 2^n, C_{n}^{0}  - C_{n}^{1}  + \cdots + (-1)^n C_{n}^{n} = 0$.
  \part $C_{n}^{k} = C_{n}^{n - k }, C_{n}^{k} = C_{n-1}^{k} + C_{n-1}^{k-1}$.
  \part $C_{n}^{k} = \frac{n}{k} C_{n-1}^{k-1}$.
  \part $C_{n}^{k} C_{k}^{m} = C_{n}^{m} C_{n-m}^{k-m} = C_{n}^{k-m} C_{n-k+m}^{m}, ~m \leq k \leq n$.
\end{parts}

\miquestion
$n$双相异的鞋共$2n$只,随机地分成$n$堆,每堆$2$只.问“各堆都自成一双鞋”这个事件$E$的概率是多少?



%force a page break for better layout of questions 
%NOTE: only force pagebreaks at the final stage for perfecting the layout
%\newpage

\miquestion
一批产品共$N$个,其中废品有$M$个,现从中随机(或者说随意)取出$n$个,问“其中恰好有$m$个废品”这个事件$E$的概率是多少?

\miquestion
$n$个男孩,$m$个女孩( $m \leq n + 1$)随机地排成一列.问“ 任意两个女
孩都不相邻”这个事件$E$的概率是多少?

\miquestion
考虑分段函数
$$
f(x) = \begin{cases}
  4Cx, &x \in (0,2); \\
  0, &x \notin (0,2)
\end{cases}
$$
试解决以下几个问题:
\begin{parts}
  \part 已知$\int_{0}^{1} f(x) dx = 1$,求解常数$C$.
  \part 求定积分
  \[ \int_{1}^{2} f(x) dx.\]
  \part 计算函数$F(x)$的表达式,$F(x)$的定义为
  \[ F(x) = \int_{-\infty}^{x} f(t) dt\]
\end{parts}

\miquestion
考虑二元函数
$$
F(x,y) = 
  1 - e^{-x} - e^{-y} + e^{-x-y-\lambda xy},
$$
其中$\lambda > 0$,试求$\lim_{x \to \infty} F(x,y), \lim_{y \to \infty} F(x,y)$.


\miquestion
计算二重积分
$$I = \iint_{D} (x^2 + 2y) dD,$$
其中$D$由$y = x^2, y = 0, x = 1$所围.
\end{questions}
\end{document}
